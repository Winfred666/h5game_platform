\documentclass[12pt]{ctexart} % 使用 ctexart 文档类支持中文

\usepackage{fancyhdr} % 奇特的 header
\usepackage{xcolor} % 更多颜色

\usepackage[utf8]{inputenc} % 支持 UTF8 字符
\usepackage{xeCJK} % 支持中文排版
\usepackage{fontspec} % 字体设置
\usepackage{geometry} % 页面布局
\usepackage{titlesec} % 自定义标题样式
\usepackage{fancyhdr} % 自定义页眉和页脚
\usepackage{setspace} % 设置行距
\usepackage{hyperref} % 超链接支持
\usepackage{tocloft} % 自定义目录样式


% 设置页面布局
\geometry{a4paper, margin=1in}

% 设置行距为 1.25 倍
\setstretch{1.25}

% 设置中文字体
\setCJKmainfont{SimSun} % 正文字体为宋体
\setCJKsansfont{SimHei} % 标题等无衬线字体为黑体
% 设置英文字体
\setmainfont{Times New Roman}


% 定义颜色
\definecolor{mygreen}{RGB}{144,238,144} % 浅绿色

% 配置页眉和页脚
\pagestyle{fancy}
\fancyhf{}
\renewcommand{\headrulewidth}{0pt}

% 左侧页眉设置, 页码在页眉外侧
\fancyhead[LE,RO]{%
  \colorbox{mygreen}{%
    \parbox[t]{1cm}{%
      \textcolor{white}{\thepage}%
    }%
  }%
}

\fancyhead[LO]{\leftmark} % 左页显示章节名
\fancyhead[RE]{\rightmark} % 右页显示节名

% 页眉横线设置

% 自定义标题样式
\titleformat{\section}
  {\normalfont\Large\bfseries\CJKfontspec{SimHei}}{\thesection}{1em}{} % 黑体标题
\titleformat{\subsection}
  {\normalfont\large\bfseries\CJKfontspec{SimHei}}{\thesubsection}{1em}{} % 黑体副标题


% 设置目录样式
\renewcommand{\cftsecfont}{\bfseries} % 目录中章节标题加粗

% 超链接设置
\hypersetup{
  colorlinks=true,
  linkcolor=blue,
  filecolor=magenta,      
  urlcolor=cyan,
}

\begin{document}

\tableofcontents % 自动生成目录

\newpage

\section{引言}
\subsection{编写目的}
该项目的目的是实现一个教学信息平台,用于教学、学习和交流。

此软件需求规格说明书描述该项目功能性需求和非功能性需求,此文档旨在为开发人员提供开发过程的参照,使开发人员能明确自己的任务以及任务完成的期限,同时也为测试人员在拟定测试用例及测试计划时提供帮助。

\subsection{项目背景}
该项目开发的软件为一个课程教学、学习平台。计算机是教育的新根,互联网则是未来的黑板。鉴于网络环境教学的普及,教育信息化已经成为教育改革发展的不可回避的趋势。教学交流网站能帮助建立新型的师生关系,更方便地把世界带入课堂。与传统的教育方式相比,网站教学使得信息量更加巨大,教学更加丰富,针对性更强。另一方面,学生需要更良好的自主学习能力,更高的信息检索和筛选能力,教师也能更及时地发布教学信息,形成新型的教学模式。

互联网是一个瞬息万变的事物,因此学生需要更加及时地了解工程活动的动态,学习更为新颖的软件工程技术,而学生与教师的交流也不能只局限于课堂交流,需要更为有效和及时的线上线下交流,因此老师提出该课程需要这样一个软件工程教学交流的网站,作为师生之间交流,获取资料的线上平台。由学生分组开发、测试,并提供上课使用。除此之外,部分没有选上该课程的同学,如果希望了解更加系统的软件工程知识,也可以通过该平台进行学习交流。

\subsection{名词定义}
% 在这里添加名词定义的内容

\section{总体描述}
\subsection{产品前景}
% 在这里添加产品前景的内容

\subsection{用户类及其特征}
% 在这里添加用户类及其特征的内容

\subsection{产品功能}
% 在这里添加产品功能的内容

\subsection{运行环境}
% 在这里添加运行环境的内容

\subsection{设计和实现上的约束}
% 在这里添加设计和实现上的约束的内容


\subsection{用户文档}
% 在这里添加用户文档的内容


\section{系统功能}
\subsection{用户需求}
% 在这里添加用户需求的内容

\subsection{用例图}
% 在这里添加用例图的内容

\subsection{功能列表}

\section{类图与 CRC 模型}
\subsection{类图}
% 在这里添加类图的内容

\subsection{CRC 模型}
\subsubsection{User}
% 在这里添加 User 的 CRC 模型


\section{非功能性需求}
\subsection{性能需求}
% 在这里添加性能需求的内容

\subsection{输入要求}
% 在这里添加输入要求的内容


\section{数据流图}

\section{验收准则}
\subsection{功能要求}
% 在这里添加功能要求的内容

\subsection{性能要求}

\subsection{存储要求}
% 在这里添加存储要求的内容

\subsection{维护要求}
% 在这里添加维护要求的内容

\section{UI 原型}
\subsection{登录界面}
% 在这里添加登录界面的内容

\subsection{注册界面}
% 在这里添加注册界面的内容

\subsection{用户主界面}
% 在这里添加用户主界面的内容

\end{document}